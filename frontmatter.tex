%
%
% UCSD Doctoral Dissertation Template
% -----------------------------------
% http:\\ucsd-thesis.googlecode.com
%
%


% REQUIRED FIELDS -- Replace with the values appropriate to you

% No symbols, formulas, superscripts, or Greek letters are allowed
% in your title.
\title{Benchmarking and Acceleration of Machine Learning and Analytics Pipelines for Large Microbiome Datasets}

\author{George Wesley Armstrong}
\degreeyear{2022}

% Master's Degree theses will NOT be formatted properly with this file.
\degreetitle{Doctor of Philosophy} 

\field{Bioinformatics and Systems Biology}
\chair{Professor Rob Knight}
% Uncomment the next line iff you have a Co-Chair
% \cochair{Professor Cochair Semimaster} 
%
% Or, uncomment the next line iff you have two equal Co-Chairs.
%\cochairs{Professor Chair Masterish}{Professor Chair Masterish}
\cochair{Professor Pieter Dorrestein}

%  The rest of the committee members  must be alphabetized by last name.
\othermembers{
Professor Vineet Bafna\\
Professor  Gal Mishne\\
Professor  Glenn Tesler\\
}
\numberofmembers{5} % |chair| + |cochair| + |othermembers|


%% START THE FRONTMATTER
%
\begin{frontmatter}

%% TITLE PAGES
%
%  This command generates the title, copyright, and signature pages.
%
\makefrontmatter 

%% DEDICATION
%
%  You have three choices here:
%    1. Use the ``dedication'' environment. 
%       Put in the text you want, and everything will be formated for 
%       you. You'll get a perfectly respectable dedication page.
%   
%
%    2. Use the ``mydedication'' environment.  If you don't like the
%       formatting of option 1, use this environment and format things
%       however you wish.
%
%    3. If you don't want a dedication, it's not required.
%
%
\begin{dedication} 
	To my parents, family, and friends.
\end{dedication}

% You are responsible for formatting here.
%\begin{mydedication} 
%  \vspace{1in}
%  \begin{flushleft}
%    To me.
%  \end{flushleft}
%   
%   \vspace{2in}
%   \begin{center}
%     And you.
%   \end{center}
%
%  \vspace{2in}
%  \begin{flushright}
%    Which equals us.
%  \end{flushright}
%\end{mydedication}



%% EPIGRAPH
%
%  The same choices that applied to the dedication apply here.
%

% The style file will position the text for you.
\begin{epigraph} 
  \emph{Computers are good at following instructions, but not at reading your mind. }\\
  ---Donald Knuth

    \vspace{2.3in}

\end{epigraph}

%% SETUP THE TABLE OF CONTENTS
%
\tableofcontents

%%
%% This block was needed to re-format the title of the glossary to match the
%% headings of the list of figures and list of tables.
%%
%% start hack:
\renewcommand{\glossarysection}[2][]{
\newpage
\noindent
\centerline{LIST OF ABBREVIATIONS}
\addcontentsline{toc}{chapter}{List of Abbreviations}
}
%% end hack

% \printglossary[title=List of Abbreviations,toctitle=List of Abbreviations,nonumberlist ]
\listoffigures  % Uncomment if you have any figures
\listoftables   % Uncomment if you have any tables

%% ACKNOWLEDGEMENTS
%
%  While technically optional, you probably have someone to thank.
%  Also, a paragraph acknowledging all coauthors and publishers (if
%  you have any) is required in the acknowledgements page and as the
%  last paragraph of text at the end of each respective chapter. See
%  the OGS Formatting Manual for more information.
%
\begin{acknowledgements}

Firstly, I would like to thank my parents, Douglas and Marjory Armstrong, for prioritizing education and doing everything within their power for over two and a half decades to empower me to achieve my goals. I would not be in this position today if it had not been for their ongoing love and support.

I would also like to recognize my loving partner, teammate, and confidant, Connie Chiang, who has unconditionally encouraged me to pursue my passions and been there to see me through some of the most difficult aspects of the last few years.

Working on a Ph. D. and figuring out what to do next has been the largest, least-structured challenge I have encountered to date. For  that reason, I would like to thank my advisor, Rob Knight for providing his expertise, rational outlook, and flexibility. This work, along with my next steps, would not have been possible without him.

Mentorship does not just come from a single source---I also received invaluable and immeasurable support from Yoshiki V\'azquez-Baeza, Cameron Martino, and Daniel McDonald. Many others I crossed paths with in the Knight Lab have also made this work possible through miscellaneous conversations, encouragement, and friendship: Gibraan Rahman, Dan Hakim, Marcus Fedarko, Imran McGrath, Kalen Cantrell, Lisa Marotz, Celeste Allaband, Justin Shaffer, Antonio Gonzalez Pena, and Shi Huang.

A doctoral dissertation is not complete without a committee---I would like to thank my committee members Gal Mishne, Glenn Tesler, Pieter Dorrestein, and Vineet Bafna for their collaboration and encouragement on this work.

Finally, I would like to extend my gratitude to many people from various stages of life who have contributed to this work influencing the path I took to get here: Ahmet Ay, Will Cipolli, Darren Strash, Michael Hay, Duke Writer, Dan Crowe, Diana Virgo, Sundaram Thirukkurungudi, Ben Apple, Tanner Gill, Ha Vu, Ben Harris, Adam Officer, Owen Chapman, and the 2018 BISB/BMI cohort.

Chapter~\ref{chapter_review}, in full, is a reprint of the material as it appears in ``Applications and Comparison of Dimensionality Reduction Methods for Microbiome Data.'' George Armstrong, Gibraan Rahman, Cameron Martino, Daniel McDonald, Antonio Gonzalez, Gal Mishne, and Rob Knight. \textit{Frontiers in Bioinformatics 2}, 2022. The dissertation author was the primary investigator and co-first author of this paper.

Chapter~\ref{chapter_faiths_pd}, in full, is a reprint of the material as it appears in ``Efficient computation of Faith's phylogenetic diversity with applications in characterizing microbiomes.'' George Armstrong, Kalen Cantrell, Shi Huang, Daniel McDonald, Niina Haiminen, Anna Paola Carrieri, Qiyun Zhu, Antonio Gonzalez, Imran McGrath, Kristen Beck, Daniel Hakim, Aki S Havulinna, Guillaume Méric, Teemu Niiranen, Leo Lahti, Veikko Salomaa, Mohit Jain, Michael Inouye, Austin D Swafford, Ho-Cheol Kim, Laxmi Parida, Yoshiki Vázquez-Baeza and Rob Knight. \textit{Genome Research 31}, 2021. The dissertation author was the primary investigator and the first author of this paper.

Chapter~\ref{chapter_umap}, in full, is a reprint of the material as it appears in ``Uniform Manifold Approximation and Projection (UMAP) Reveals Composite Patterns and Resolves Visualization Artifacts in Microbiome Data.'' George Armstrong, Cameron Martino, Gibraan Rahman, Antonio Gonzalez, Yoshiki Vázquez-Baeza, Gal Mishne, and Rob Knight.  \textit{mSystems 6}, 2021. The dissertation author was the primary investigator and the first author of this paper.

Chapter~\ref{chapter_host_filtering}, in full, is a reprint of the material as it appears in ``Swapping metagenomics preprocessing pipeline components offers speed and sensitivity increases.'' George Armstrong, Cameron Martino, Justin Morris, Behnam Khaleghi, Jaeyoung Kang, Jeff DeReus, Qiyun Zhu, Daniel Roush, Daniel McDonald, Antonio Gonzalez, Justin Shaffer, Carolina Carpenter, Mehrbod Estaki, Stephen Wandro, Sean Eilert, Ameen Akel, Justin Eno, Ken Curewitz, Austin D. Swafford, Niema Moshiri, Tajana Rosing, and Rob Knight. \textit{mSystems e0137821}, 2022. The dissertation author was a primary investigator and co-first author of this paper.


\end{acknowledgements}


%% VITA
%
%  A brief vita is required in a doctoral thesis. See the OGS
%  Formatting Manual for more information.
%
\begin{vitapage}
\begin{vita}
  \item[2014--2018] B.~A. in Mathematics, Colgate University
  \item[2021] Software Engineering Intern, Thermo Fisher Scientific
  \item[2021--2022] Bioinformatics Artificial Intelligence Intern, NVIDIA Corporation
  \item[2018--2022] Ph.~D. in Bioinformatics and Systems Biology, University of California San Diego
\end{vita}


\begin{publications}

    \item \textsl{Author names marked with $\dagger$ indicate shared first co-authorship}.

    \item \textbf{George Armstrong}$\dagger$, Gibraan Rahman$\dagger$, Cameron Martino, Daniel McDonald, Antonio Gonzalez, Gal Mishne, and Rob Knight. ``Applications and Comparison of Dimensionality Reduction Methods for Microbiome Data.'' \textit{Frontiers in Bioinformatics 2}, 2022.
    
    \item \textbf{George Armstrong}, Kalen Cantrell, Shi Huang, Daniel McDonald, Niina Haiminen, Anna Paola Carrieri, Qiyun Zhu, Antonio Gonzalez, Imran McGrath, Kristen Beck, Daniel Hakim, Aki S Havulinna, Guillaume M\'eric, Teemu Niiranen, Leo Lahti, Veikko Salomaa, Mohit Jain, Michael Inouye, Austin D Swafford, Ho-Cheol Kim, Laxmi Parida, Yoshiki V\'azquez-Baeza and Rob Knight. ``Efficient computation of Faith's phylogenetic diversity with applications in characterizing microbiomes.'' \textit{Genome Research 31}, 2021.
    
    \item \textbf{George Armstrong}, Cameron Martino, Gibraan Rahman, Antonio Gonzalez, Yoshiki V\'azquez-Baeza, Gal Mishne, and Rob Knight. ``Uniform Manifold Approximation and Projection (UMAP) Reveals Composite Patterns and Resolves Visualization Artifacts in Microbiome Data.'' \textit{mSystems 6}, 2021.
    
    \item \textbf{George Armstrong}$\dagger$, Cameron Martino$\dagger$, Justin Morris, Behnam Khaleghi, Jaeyoung Kang, Jeff DeReus, Qiyun Zhu, Daniel Roush, Daniel McDonald, Antonio Gonzalez, Justin Shaffer, Carolina Carpenter, Mehrbod Estaki, Stephen Wandro, Sean Eilert, Ameen Akel, Justin Eno, Ken Curewitz, Austin D. Swafford, Niema Moshiri, Tajana Rosing, and Rob Knight. `Swapping metagenomics preprocessing pipeline components offers speed and sensitivity increases.''  \textit{mSystems e0137821}, 2022.

    \item \noindent\rule[0.5ex]{\linewidth}{0.5pt}

    \textsl{The following publications were not included as part of this dissertation, but were also significant byproducts of my doctoral training.}

    \item Cameron Martino, Liat Shenhav, Clarisse A Marotz, \textbf{George Armstrong}, Daniel McDonald, Yoshiki V\'azquez-Baeza, James T Morton, Lingjing Jiang, Maria Gloria Dominguez-Bello, Austin D Swafford, Eran Halperin, Rob Knight. ``Context-aware dimensionality reduction deconvolutes gut microbial community dynamics.'' \textit{Nature biotechnology 39}, 2021.
    
    \item  Kalen Cantrell, Marcus W. Fedarko, Gibraan Rahman, Daniel McDonald, Yimeng Yang, Thant Zaw, Antonio Gonzalez, Stefan Janssen, Mehrbod Estaki, Niina Haiminen, Kristen L. Beck https://orcid.org/0000-0002-4603-0235, Qiyun Zhu, Erfan Sayyari, James T. Morton, \textbf{George Armstrong}, Anupriya Tripathi, Julia M. Gauglitz, Clarisse Marotz, Nathaniel L. Matteson, Cameron Martino, Jon G. Sanders, Anna Paola Carrieri, Se Jin Song, Austin D. Swafford, Pieter C. Dorrestein, Kristian G. Andersen, Laxmi Parida, Ho-Cheol Kim, Yoshiki Vázquez-Baeza, Rob Knight. ``EMPress Enables Tree-Guided, Interactive, and Exploratory Analyses of Multi-omic Data Sets.'' \textit{mSystems e01216-20}, 2021.
    
    \item Qiyun Zhu, Shi Huang, Antonio Gonzalez, Imran McGrath, Daniel McDonald, Niina Haiminen, \textbf{George Armstrong}, Yoshiki Vázquez-Baeza, Julian Yu, Justin Kuczynski, Gregory D. Sepich-Poore, Austin D. Swafford, Promi Das, Justin P. Shaffer, Franck Lejzerowicz, Pedro Belda-Ferre, Aki S. Havulinna, Guillaume Méric, Teemu Niiranen, Leo Lahti, Veikko Salomaa, Ho-Cheol Kim, Mohit Jain, Michael Inouye, Jack A. Gilbert, Rob Knight. ``Phylogeny-Aware Analysis of Metagenome Community Ecology Based on Matched Reference Genomes while Bypassing Taxonomy.'' \textit{mSystems e00167-22}, 2022.
    
    \item Clarisse Marotz, Pedro Belda-Ferre, Farhana Ali, Promi Das, Shi Huang, Kalen Cantrell, Lingjing Jiang, Cameron Martino, Rachel E Diner, Gibraan Rahman, Daniel McDonald, \textbf{George Armstrong}, Sho Kodera, Sonya Donato, Gertrude Ecklu-Mensah, Neil Gottel, Mariana C Salas Garcia, Leslie Y Chiang, Rodolfo A Salido, Justin P Shaffer, Karenina Sanders, Greg Humphrey, Gail Ackermann, Niina Haiminen, Kristen L Beck, Ho-Cheol Kim, Anna Paola Carrieri, Laxmi Parida, Yoshiki Vázquez-Baeza, Francesca J Torriani, Rob Knight, Jack Gilbert, Daniel A Sweeney, Sarah M Allard. ``SARS-CoV-2 detection status associates with bacterial community composition in patients and the hospital environment.'' \textit{Microbiome 9}, 2021.
    
    \item Igor Sfiligoi, \textbf{George Armstrong}, Antonio González, Daniel McDonald, Rob Knight. ``Optimizing UniFrac with OpenACC yields \>1000x speed increase''. \textit{Under Review at mSystems}, 2022.
    
    \item  Cameron Martino, Daniel McDonald, Kalen Cantrell, Amanda Hazel Dilmore, Yoshiki Vázquez-Baeza, Liat Shenhav, Justin P. Shaffer, Gibraan Rahman, \textbf{George Armstrong}, Celeste Allaband, Se Jin Song, Rob Knight. ``Compositionally aware phylogenetic beta-diversity measures better resolve microbiomes associated with phenotype.'' \textit{In Press at mSystems}, 2022.
    

\end{publications}

\end{vitapage}


%% ABSTRACT
%
%  Doctoral dissertation abstracts should not exceed 350 words. 
%   The abstract may continue to a second page if necessary.
%
\begin{abstract}

Within the past decade, the number of publicly available microbiome sequencing samples has increased dramatically. Consequently, bottlenecks have arisen in common analysis steps, such as processing the sequencing data and characterizing the content of the microbial communities. Over this timespan, new tools have also been developed for steps such as alignment and dimensionality reduction that scale better or handle the additional complexity of high-dimensional data, however, their characteristics on microbiome data were previously uncharacterized. In this thesis, we accelerate the analysis of microbiomes by introducing new methods or benchmarking alternatives. Additionally, we compare the results of novel methodology to existing best-practices on gold-standard datasets to determine whether the methods adequately address the specific challenges of microbiome data.

In the first part of this thesis, Chapter 1 reviews many aspects of microbiome data that necessitate the use of microbiome-specific techniques for analyzing collections of microbial communities. Chapter 2 then introduces SFPhD, a novel approach for calculating phylogenetic alpha diversity that leverages the characteristics of microbiome data to speed up and reduce the memory requirements of a costly single-sample characterization.

In the second part of the thesis, we apply recently developed tools for machine learning and sequencing pre-processing to demonstrate their potential for elucidating complex relationships in microbial data and reducing the lead time for supporting clinical applications of metagenomic sequencing, respectively. Chapter 3 demonstrates how Uniform Manifold Approximation and Projection (UMAP) provides succinct representations of data compared to the long-time standard method of microbial ecology, Principal Coordinates Analysis (PCoA). Importantly, UMAP provides different guarantees about the preservation of local/global geometry in its representation and careful consideration should be given to its application. In Chapter 4, we show that the \textit{de facto} metagenomic preprocessing pipeline of Atropos for adapter trimming and Bowtie2 for host filtering can be replaced by a substantially faster combination of Fastp and Minimap2, respectively. Furthermore, we have determined that the results this new pipeline produces are qualitatively comparable to the outputs produced by the original pipeline.

\end{abstract}


\end{frontmatter}
